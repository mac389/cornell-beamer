\input{config/commands}
\input{config/hyphenation}

\setbeamertemplate{Cornell}[numbered]
\usepackage{booktabs, graphicx,ulem}
\begin{document}

%\AtBeginSection[]
%{
%  \frame<handout:0>
%  {
%    \frametitle{Outline}
%    \tableofcontents[currentsection,hideallsubsections]
%  }
%}
%
%\AtBeginSubsection[]
%{
%  \frame<handout:0>
%  {
%    \frametitle{Outline}
%    \tableofcontents[sectionstyle=show/hide,subsectionstyle=show/shaded/hide,subsubsectionstyle=hide]
%  }
%}
%
%\AtBeginSubsubsection[]
%{
%  \frame<handout:0>
%  {
%    \frametitle{Outline}
%    \tableofcontents[sectionstyle=show/hide,subsectionstyle=show/shaded/hide,subsubsectionstyle=show/shaded/hide]
%  }
%}


\frame{\titlepage}
\newcommand<>{\highlighton}[1]{%
  \alt#2{\structure{#1}}{{#1}}
}

\newcommand{\icon}[1]{\pgfimage[height=1em]{#1}}

\section*{}
\begin{frame}{Content}
\tableofcontents
\end{frame}

%%%%%%%%%%%%%%%%%%%%%%%%%%%%%%%%%%%%%%%%%
%%%%%%%%%% Content starts here %%%%%%%%%%
%%%%%%%%%%%%%%%%%%%%%%%%%%%%%%%%%%%%%%%%%

\section{Intro}
\begin{frame}
\frametitle{Report Card}



\resizebox{\columnwidth}{!}{%
\begin{tabular}{@{}lllll@{}}
\toprule
Project &
  Milestones &
  Difficulties &
  Funded &
  Collaboration \\ \midrule
Identify novel opioid addiction treatments from social media &
  Nothing published &
   &
  Topic of DP5, CDRMP &
   \\
    & & & & \\
Estimating sublethal 2,4-DNP use from social media &
  Nothing published &
  In between CS and medicine &
   &
   \\
      & & & & \\
\begin{tabular}[c]{@{}l@{}}Developing a clinical decision tool to identify \\ non-canonical presentations of opioid agonist toxicity\end{tabular} &
  Nothing published &
   &
   & Habboushe?
   \\
   & & & & \\
\begin{tabular}[c]{@{}l@{}}Developing a neural network to classify the structure of \\ novel psychoactive substances as  primarily (opioid, \\ phenethylamine, tryptamine, cannabinoid, or benzodiazepine)\end{tabular} &
  Nothing published &
   &
   &
   \\
      & & & & \\
\begin{tabular}[c]{@{}l@{}}Identify previously unknown adverse effects \\ from COVID-19 therapeutics\end{tabular} &
  \begin{tabular}[c]{@{}l@{}}1 review published, \\ 2 pre-accepted\end{tabular} &
  \begin{tabular}[c]{@{}l@{}}ACMT data collection not rigorous\\ enough to allow for original research\end{tabular} &
  Yes, by FDA &
  FDA, ACMT \\
     & & & & \\
Developing an ontology of pediatric poisoning &
  Nothing Published &
  Bandwidwth &
  No &
  Yes, Lyon \\ \bottomrule
\end{tabular}%
}


\end{frame}

\begin{frame}{Who I work with}
\settowidth{\leftmargini}{\usebeamertemplate{itemize item}}
\addtolength{\leftmargini}{\labelsep}
\begin{itemize}
	\item[] Andrew Moreno (WCM Class of 24, AOC): Decision Tool
	\item[]
	\item[] Abdel Abdelati (NYP-Qatar, Class of 24): DNP
	\item[] 
	\item[] Ciara Balanz\'{a} (High School): Neural Network
	\item[] 
	\item[] Stefan Bartell, PhD: Drug Discovery, Ontology
\end{itemize}

\end{frame}

\section[Drug Discovery]{Novel addiction treatments from social media}

\begin{frame}{In One Page}
\settowidth{\leftmargini}{\usebeamertemplate{itemize item}}
\addtolength{\leftmargini}{\labelsep}
\begin{itemize}
	\item[] \uline{Background}: Rising OUD, marginally ineffective treatments
	\item[] \uline{Insight}: Online Discussions are a Proxy for Self-Experimentation
	\item[] \uline{Hypothesis}: Online Discussions May Identify Druggable Lead Compounds
	\item[] \uline{Approach}
	\begin{itemize}
		\item[] NLP of Social Media
		\item[] Link social media data with established effects via ontology
	\end{itemize} 
	\item[] \uline{Work To Date}
	\item[] \uline{Next Steps}
\end{itemize}
\end{frame}

\begin{frame}{Preliminary Data}

\end{frame}

\section{2,4-Dinitrophenol}

\begin{frame}{In One Page}
\settowidth{\leftmargini}{\usebeamertemplate{itemize item}}
\addtolength{\leftmargini}{\labelsep}
\begin{itemize}
	\item[] \uline{Background}: Disordered eating; alluring supplement used despite FDA ban
	\item[] \uline{Insight}: Online Discussions are the only \emph{human} data source
	\item[] \uline{Hypothesis}: Online Discussions May Identify Dose-Effect Relationships
	\item[] \uline{Approach}
	\begin{itemize}
		\item[] NLP of Social Media
		\item[] Link social media data with established effects via ontology
	\end{itemize} 
	\item[] \uline{Work To Date}
	\item[] \uline{Next Steps}
\end{itemize}
\end{frame}

\section[Decision Tool]{Clinical Decision Tool for Novel Synthetic Opioids}
\begin{frame}{In One Page}
\settowidth{\leftmargini}{\usebeamertemplate{itemize item}}
\addtolength{\leftmargini}{\labelsep}
\begin{itemize}
	\item[] \uline{Background}: New opioids (seem to) cause more toxicity than traditional toxidrome
	\item[] \uline{Insight}: Ontologies can support MDM more transparently than other approaches
	\item[] \uline{Approach}
	\begin{itemize}
		\item[] Ontology of Novel Synthetic Opioids
		\item[] Software Development (MD Calc integration?)
	\end{itemize} 
	\item[] \uline{Work To Date}
	\item[] \uline{Next Steps}
\end{itemize}
\end{frame}


\section[Neural Network]{Predicting SAR of Novel Substances}
\begin{frame}{In One Page}
\settowidth{\leftmargini}{\usebeamertemplate{itemize item}}
\addtolength{\leftmargini}{\labelsep}
\begin{itemize}
	\item[] \uline{Background}: People describe self-treatment online
	\item[] \uline{Insight}:Online Discussions May Identify Druggable Lead Compounds 
	\item[] \uline{Approach:} NLP $\Rightarrow$ CNN $\Rightarrow$ Ontology
	\item[] \uline{Work To Date}
	\item[] \uline{Next Steps}
\end{itemize}
\end{frame}

\section[COVID]{Identifying rare adverse events from COVID Therapeutics}
\begin{frame}{In One Page}

The FDA has contracted with ACMT to identify adverse events from COVID therapeutics (multi-site data collection)\\
\vspace{1em}
We are reporting.\\
\vspace{1em}
Just started screening specific populations (pregnant, pediatric). \\
\end{frame} 


\section[Les enfants]{Apply AI to Pediatric Poisoning}
\begin{frame}{In One Page}
\settowidth{\leftmargini}{\usebeamertemplate{itemize item}}
\addtolength{\leftmargini}{\labelsep}
\begin{itemize}
	\item[] \uline{Background}: AI can identify the causes of some poisonings in adults. Children aren't small adults. 
	\item[] \uline{Insight}: Markov logic can extend ontologies
	\item[] \uline{Approach:} Clinical Data $\Rightarrow$ Markov Coefficients; Ontology
	\item[] \uline{Work To Date}
	\item[] \uline{Next Steps}
\end{itemize}
\end{frame}

\begin{frame}{What's an Ontology?}

\end{frame}

\section{Queens Research}
\begin{frame}{Tracking Grid}
\resizebox{\columnwidth}{!}{%
\begin{tabular}{@{}lllll@{}}
\toprule
Project &
  Milestones &
  Difficulties &
  Funded &
  Collaboration \\ \midrule
 Effect of Translator Use on TTD, Pain Medication &
  In Revision &
   & &
   \\
    & & & & \\
Digital Phenotypes of DKA by Dispo &
  Applying for IRB &
   &
   & Yiye?
   \\
      & & & & \\
 RCT of proning in mild COVID &
  Writing manuscript &
   &
   &
   \\
   & & & & \\
  \bottomrule
\end{tabular}%
}

\end{frame}

%\section[Future]{Goals for the next year}
%\begin{frame}
%\frametitle{References}
%References can be used with the \texttt{BibTeX} commands  \cite{Knuth.1986}. The list of references  will be shown at the end of the presentation with the preferred style.
%\end{frame}

%%%%%%%%%%%%%%%%%%%%%%%%%%%%%%%%%%%%%%%%%
%%%%%%%%%%       References      %%%%%%%%
%%%%%%%%%%%%%%%%%%%%%%%%%%%%%%%%%%%%%%%%%

%\section*{}
%\begin{frame}[allowframebreaks]{References}
%\def\newblock{\hskip .11em plus .33em minus .07em}
%\scriptsize
%\bibliographystyle{alpha}
%\bibliography{literature/bib}
%\normalsize
%\end{frame}

\end{document}